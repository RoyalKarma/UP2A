\section{Kyberkriminalita a elektronické důkazy}

\subsection{Popište nároky kladené na zákonnost elektronického důkazu.}
Elektronický důkaz = jakýkoliv důkaz přenášen v digitální podobě. Emaily, dig. fotografie, IM historie, textové dokumenty, video a audio záznamy, provozní údaje ...

Nároky na zákonnost:\begin{itemize}
    \item Důkaz byl opatřen způsobem, který stanovuje/připouští zákon
    \item Důkaz opatřen a proveden oprávněným procesním subjektem
    \item Neoprávněné získání - absolutní(neodstranitelná)/ relativní(odstranitelná) neúčinnost
\end{itemize}
Legální evidence na základě nelegálního důkazu je nepřípustná\\
Distributivní x nedistributivní právo
\begin{itemize}
    \item Standard dokazování: \uv{Nejsou důvodné pochybnosti}
    \item Na zákonnost důkazů se pokoušel k ÚS odvolávat i David Rath. Proti němu byly použity odposlechy a proto byl chycen při činu, odvolával se na nezákonnost těchto důkazů, čímž by padly vvšechny důkazy na nich stojící, protože všechny by byly nelegálně získané. Nezákonnost odposlechů ale ÚS zamítl, Rath odsouzen byl.
    \item Elektronické důkazy mají velkou výtěžnost
    \item With great power comes great responsibility, Spiderman
    \item Důraz kladený na zákonnost je z toho důvodu, abychom nežili v 1948, kde by byl každý sledován z preventivních důvodů
    \item Na závažnější zásahy do soukromí je potřeba soudní příkaz (odposlech) nebo povolení státního zástupce (sledování), to má sloužit proti zneužívání práva, ale bohužel se tyto žádosti nekontrolují tak důkladně, jak by měly
    \item Aby byl odposlech legální, je potřeba k němu vést protokol
\end{itemize}


\subsection{Vysvětlete strukturu §230 TZ (Neoprávněný přístup k počítačovému systému a nosiči informací). Uveďte příklady k jednotlivým odstavcům.}
\begin{itemize}
    \item Odstavec 1
          \begin{itemize}
              \item \textit{\uv{Kdo překoná bezpečnostní opatření, a tím neoprávněně získá přístup k počítačovému systému nebo k jeho části...}}
              \item V podstatě se jedná o prolomení důvěrnosti (CIA triáda = důvěrnost, integrita, dostupnost)
              \item Počítačový systém je chápán velmi široce (server, PC, kamera, telefon, router, webkamera...)
              \item Nelegální je prolomení jakéhokoli bezpečnostního opatření a nezáleží na síle bezpečnostního opatření (heslo, firewall, šifrování)
              \item Příklad: Použiju WiFi password cracker a v tu chvíli mě švestky berou
          \end{itemize}
    \item Odstavec 2
          \begin{itemize}
              \item \textit{\uv{Kdo získá přístup a data neoprávněně užije NEBO data neoprávněně vymaže, zničí, poškodí, změní, potlačí, sníží jejich kvalitu NEBO data padělá a pozmění tak, aby byla považována za pravá NEBO neoprávněně vloží data do počítačového systému...}}
              \item Jedná se o prolomení integrity a/anebo dostupnosti z CIA
              \item Neřeší se, jestli bylo překonáno opatření, za toto může být stíhán, i člověk, který měl oprávněný přístup, ale zneužil ho ke škodě
              \item Příklad: Jsem ajťák, šéf mi řekl, že na konci měsíce končím, tak mu složím databázi a už mě švestky berou
          \end{itemize}
    \item Odstavce 3-5 udávají výši trestů
          \begin{itemize}
              \item Všechny tresty jsou odnětí svobody
              \item Přitěžující okolnosti jsou rozsah škody, vlastní zisk, přítomnost v organizované skupině nebo pokud je takový útok specificky mířený na podnik nebo veřejnou správu
          \end{itemize}
\end{itemize}


\subsection{Popište rozdíly ve fungování §88 TŘ a §8 odst. 5 TŘ ve vztahu k mobilnímu telefonu.}
\begin{itemize}
    \item §8 odst. 5 - povinnost součinností
          \begin{itemize}
              \item Požadujeme sdělení informací od zprostředkovatelské služby - Státní orgány...
              \item můžeme požadovat logy, metadata
              \item Toto právo lze vynucovat soudem
              \item lze také ověřit hovory a SMS v telefonu, který byl policií již zajištěn, ale ne zprávy, které přicházejí až po zabavení, na ty už je potřeba aplikovat odposlech\\
          \end{itemize}

    \item §88 TŘ
          \begin{itemize}
              \item Může být vydán příkaz k odposlechu a záznamu telekomunikačního provozu, pokud lze důvodně předpokládat, že jím budou získány významné skutečnosti pro trestní řízení a nelze-li sledovaného účelu dosáhnout jinak nebo bylo-li by jinak jeho dosažení podstatně ztížené
          \end{itemize}
    \item Rozdíly
          \begin{itemize}
              \item §88 TŘ - menší pravděpodobnost porušení práv
              \item §8 v případě potřeby pouze metadat a přímého kontaktu s poskytovatelem
          \end{itemize}
\end{itemize}


\subsection{Vysvětlete, jaké elektronické důkazy mohou hrát roli při vyšetřování trestného činu vraždy podle §140 TZ.}
\begin{itemize}
    \item Máme materiální a formální pravdu. Důkazy slouží k nalezení formální pravdy, materiální pravda je nedosažitelná, pokud to vyšetřovatel neviděl na vlastní oči, protože neumíme cestovat časem
    \item Práce s důkazy
          \begin{itemize}
              \item Volné hodnocení důkazů - Soudce sám přisuzuje důkazům váhu a pravdivost
              \item Zákonná teorie důkazní - Rigorózní pravidla
          \end{itemize}
    \item Důkazy shromažďují orgány činné v trestním řízení (OČTŘ)
    \item Shromažďování důkazů
          \begin{itemize}
              \item Součinnost (Sdělte nám hento a tento)
              \item Povinnost k vydání věci (Hmotná věc, ale data následují osud věci, takže telefon i s daty)
              \item Odposlech (Vztahuje se např. i na data, která přijdou na telefon až po jeho zabavení, potřeba soudní příkaz nebo povolení účastníka odposlouchávané stanice)
              \item Záznam telekomunikačního provozu (Metadata, tzv. Data Retention, náhrady provozovatelům sítě)
              \item Sledování osob a věcí (Povolení státního zástupce, možnost při sledování dělat záznamy), nelze použít na telekomunikační provoz, ale jen na uložená data
          \end{itemize}
    \item Každý důkaz musí být pořízen přiměřeně k závažnosti trestného činu, protože se jedná o zásah do soukromí. Při vraždě by ale měly orgány mít možnost využít včech těchto 5 paragrafů.
\end{itemize}


\subsection{Popište, co je to data retention a vysvětlete zásadní milníky v jeho legislativní historii (rozhodovací praxe SDEU a ÚS).}

Pojem data retention označuje ukládání provozních a lokalizačních údajů u poskytovatelů telekomunikačních služeb, převážně pro účely vyšetřování trestné činnosti.\\
Zákon č. 127/2005 Sb., o elektronických komunikacích v § 97, odst. 3
\begin{itemize}
    \item Založen evropskou směrnicí (2006/24/ES)
    \item SDEU opakovaně judikoval, že plošné sledování odporuje evropskému právu i Chartě základních práv EU
    \item ÚS zrušil i u nás, zrušení povinnosti ukládání (Zákon o elektronických komunikacích a Trestní řád dostaly novelu, opravdu zrušeno)
    \item Poté nová úprava, data retention opět zavedeno, ale metadata přístupná jen pro některé trestné činy a uchováváno na kratší dobu
    \item ÚS nakonec došel k závěru, že česká právní úprava, která uchovávání dat přikazuje, je v pořádku
\end{itemize}
