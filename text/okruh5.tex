\section{Ochrana osobních údajů}

\subsection{Ochrana soukromí a ochrana osobních údajů -- popište vztah mezi právní úpravou ochrany soukromí a osobních údajů}
\begin{itemize}
    \item Osobní údaj = informace týkající se určitého subjektu (fyzická osoba), která ho popisuje (rodné číslo, datum narození, identita)
    \begin{itemize}
        \item ale také i kombinace osobních a neosobních údajů - IP adresa
    \end{itemize}
    \item Soukromí = právo nechán být na pokoji (Není nějak extra dobře definované)
    \item Rozdílné cíle a účely úpravy
    \begin{itemize}
        \item Ochrana soukromí: občanský zákoník, při porušení se podává žaloba
        \item Ochrana os. údajů: GDPR (-> Zákon o zpracování osobních údajů), ochrana před zásahem a korektivní zpracování
    \end{itemize}
    \item Reaktivní princip (u ochrany soukromí) vs. preventivní princip (os. úd.)
    \item Občanské právo (soukromí) vs. správní právo (os. úd.)
\end{itemize}


\subsection{Základní principy a zásady zpracování osobních údajů}
\begin{itemize}
    \item GDPR
    \item Zpracování = jakékoli nakládání s údaji (široká aplikace)
    \item Správce os. úd. určuje účely a prostředky zpracování
    \item Zpracovatel je ten, kdo je pověřen správcem, musí mít smlouvu a oprávnění, zpracovatel může a nemusí být, může si to dělat správcem sám
    \item Accountability (Performativní pravidlo) \uv{Nastav své zpracování tak, aby odpovídalo potřebám konkrétní situace, abys zpracovával osobní údaje korektně a férově.}
    \item Přístup založený na riziku (Tetička ve večerce nemusí mít stejnou úroveň ochrany údajů svých zákazníků jako Google)
    \item Je důležité mít účel zpracování a jen pro tento účel lze údaje použít
    \begin{itemize}
        \item Ne moc konkrétní, protože pak je nutno hned smazat po splnění účelu a pro jiné, podobné účely by se údaje nesměly využít
        \item Ne moc abstraktní, protože to před soudem neobstojí
    \end{itemize}
    \item Zásady
    \begin{itemize}
        \item Osobní údaje musí být ve vztahu k subjektu údajů zpracovávány \textbf{korektně} a zákonným a \textbf{transparentním} způsobem
        \item Zásada \textbf{limitace účelem}, minimalizace údajů, přesnosti, omezení uložení, integrity a důvěrnosti, odpovědnosti
        \item Povinnosti správce - Stanoví účel, prostředky, způsob zpracování, musí dodžení dokázat doložit
    \end{itemize}
    \item Zákonnost zpracování: je potřeba, aby bylo splněno alespoň jeden z právních titulů:
    \begin{itemize}
        \item Souhlas
        \item Zpracování nezbytné pro plnění smlouvy
        \item Zpracování pro dodržení právní povinnosti správce
        \item Ochrana životně důležitých zájmů subjektu
        \item Zpracování pro plnění úkolu ve veřejném zájmu
        \item Zpracování pro ochranu právem chráněných zájmů správce nebo jiné osoby (test proporcionality)
    \end{itemize}
\end{itemize}


\subsection{Osobní údaje a jejich zpracování -- vymezení pojmu a zúčastněné osoby}
\begin{itemize}
    \item Osobní údaje
    \begin{itemize}
        \item Veškeré informace o identifikované nebo identifikovatelné fyzické osobě
        \item Objektivní pojetí - existuje (byť teoretická) možnost člověka identifikovat z těchto dat, je to osobní údaj
        \item Subjektivní pojetí - Existuje dosažitelná a legální metoda, jak člověka z dat identifikovat? Ano -> Je to osobní údaj.
        \item Anonymizace nebo pseudonymizace mohou být řešením, ale u pseudonymizace stále jde o osobní údaje
    \end{itemize}
    \item Zpracování osobních údajů
    \begin{itemize}
        \item Jakákoliv operace nebo soubor operací s osobními údaji
    \end{itemize}
    \item Zúčastněné osoby:
    \begin{itemize}
        \item Správce - Určuje účely a prostředky zpracování osobních údajů
        \item Zpracovatel - Jeho jmenuje správce, má se správcem smlouvu, musí mít oprávnění a pověření
        \item Subjekt - Jeho údaje jsou zpracovávány
    \end{itemize}
\end{itemize}

\subsection{Povinnosti správce osobních údajů podle nařízení 2016/679}
\begin{itemize}
    \item Jmenovanému nařízení se říká GDPR
    \item Povinnnosti:
    \begin{itemize}
        \item Odpovědnost správce
        \item Záměrná a standardní ochrana osobních údajů
        \item Záznamy o činnostech zpracování
        \item Spolupráce s dozorovým úřadem
        \item Zabezpečení zpracování
        \item Ohlašování případů porušení zabezpečení osobních údajů dozorovému úřadu
        \item Posouzení vlivu na ochranu osobních údajů
        \begin{itemize}
            \item Pokud se předpokládá vysoké riziko pro práva a svobody osob
            \item Povinně při profilování, rozsáhlém zpracování citlivých údajů, rozsáhlém monitorování veřejně přístupných prostor
            \item 9 kategorií, ve kterých se vliv posuzuje
        \end{itemize}
        \item Jmenování pověřence pro ochranu osobních údajů
        \begin{itemize}
            \item Znalec práva (nejen)
            \item Ve speciálních případech je nutné ho jmenovat
            \item Tvoří kontakt úřadu a správce
        \end{itemize}
    \end{itemize}
\end{itemize}


\subsection{Práva subjektu údajů ve vztahu ke správci a zpracovateli podle nařízení 2016/679}
\begin{itemize}
    \item Google Spain souvisí s právem být zapomenut, šlo o člověka, o kterém Google našel staré články o exekuci, SDEU usoudil, že Google musí link vymazat, je to osobní informace
    \item Právo být informován o zpracování osobních údajů
    \begin{itemize}
        \item Základní informace o zpracování (Kdo, proč (účel a právní titul), jak dlouho, kde...)
    \end{itemize}
    \item Právo na přístup k údajům (Poskytnutí kopie údajů)
    \item Právo na opravu
    \item Právo na výmaz (\uv{právo být zapomenut}, virální povaha (všichni, jimž byly údaje poskytnuty, je musí smazat)), Aplikovatelné, když:
    \begin{itemize}
        \item osobní údaje již nejsou potřebné pro účely, pro které byly shromážděny nebo jinak zpracovány
        \item subjekt údajů odvolá souhlas a neexistuje žádný další právní důvod pro zpracování
        \item subjekt údajů vznese námitky proti zpracování 
        \item osobní údaje byly zpracovány protiprávně
        \item osobní údaje musí být vymazány ke splnění právní povinnosti stanovené v právu Unie nebo členského státu, které se na správce vztahuje
        \item osobní údaje byly shromážděny v souvislosti s nabídkou služeb informační společnosti
    \end{itemize}
    \item Právo na omezení zpracování 
    \item Právo na přenositelnost údajů 
    \item Právo vznést námitku (Když zpracování z důvodu: oprávněného zájmu NEBO plnění úkolu ve veřejném zájmu NEBO přímý marketing)
    \item Právo na ochranu před automatizovaným individuálním rozhodování, včetně profilování 
\end{itemize}
