\section{Výběr z přednášek}

\subsection{Předmět a účel právní regulace elektronických komunikací}
\begin{itemize}
    \item předměty regulace:
    \begin{itemize}
        \item Zajištění sítí el. komunikací
        \item Poskytnutí služeb el. komunikací
        \item provoz přístrojů
    \end{itemize}
    \item účely:
    \begin{itemize}
        \item Zajištění funkční hospodářské soutěže, konkurence a ochrany spotřebitelů
        \item Síťová neutralita
        \item Modernizace infrastruktury
        \item Ochrana soukromí a osobních údajů (ePrivacy
    \end{itemize}
\end{itemize}


\subsection{Síťová neutralita -- popis, právní úprava v EU, aktuální výzvy v USA}
\begin{itemize}
    \item Rovné podmínky přístupu k obsahu
    \item práv. úprava EU: nařízení č. 2015/2120 - umožněn zero rating (bezplatný přístup k internetu za určitch podmínek
    \item USA : \begin{itemize}
        \item  FCC (Federal communications commission) odvolala pravidlo které zaručuje net neutrality
        \item spor Mozilla vs FCC (2/2018)
        \item 9/2018 Přijetí California Internet Consumer Protection and Net Neutrality
        \item 10/2019 - rozhodnutí federálního odvolacího soudu - FCC mohla zrušit net neutrality, ale jednotlivé státy si mohou přijmout vlastní legislativu
        \item 7/2021 Biden doporučil FCC obnovení pravidel zajišťující net neutrality
    \end{itemize}
\end{itemize}


\subsection{Základní registry -- popis, účel a právní úprava}
\begin{itemize}
    \item  Účelem je zefektivnění a využítí možností současných technologií pro online přístupy kdykoliv a odkudkolive
    \item Omezení nezbytného sběru dat a informací a jejich efektivní předávání
    \item Zákonná úprava z. č. 111/2009 Sb., o základních registrech
    \item 4 základní:
    \begin{itemize}
        \item registr obyvatel\begin{itemize}
            \item referenční údaje o fyzických osobách
            \item spravuje ministerstvo vnitra a PČR
            \item info o občanech ČR, cizincích s trvalým pobytem, cizinci, jimž byl udělejn azyl, jiné fyz. osoby u nichž zákon stanovuju že budou uvedeny v registru obyvatel
        \end{itemize} 
        \item registr osob \begin{itemize}
            \item  evidence právnických osob, organizačních složek státu, organizací s mezinárodním prvkem, podnikajících fyzických osob
            \item Obsahuje základní údaje o osobách
            \item využívají všechny orgány veřejné správy, které mají k tomuto oprávnění
        \end{itemize}
        \item registr územní identifikace\begin{itemize}
            \item Katastrální úřad, Stavební úřad, Obce
            \item info o územních prvcích, adresách, územních identifikací, územně evidenčních jednotkách
        \end{itemize}
        \item registr práv a povinností \begin{itemize}
            \item Ministerstvo vnitra
            \item Zdroj údajů pro informační systémy zákl. registrů při řízení přístupu uživatelů k údajům v jednotlivých registrech
            \item referenční údaje o právech a povinnostech osob
        \end{itemize}
    \end{itemize}
\end{itemize}

\subsection{Hlavní překážky pro zavedení komplexní eJustice v ČR}
\begin{itemize}
    \item  elektronizace soudnictví, využítí ICT v justici
    \item problémy
    \begin{itemize}
        \item Odpor justičního personálu
        \item Stereotypy, návyky
        \item Nekoncepčnost
        \item Zabezpečení
        \item Kompatibilita

    \end{itemize}
\end{itemize}

\subsection{Co je to princip distinkce a jak se vztahuje ke kyberzbraním a kyber-válce?}
\begin{itemize}
    \item Mezinárodní právo
    \item Problém dělání rozdílů mezi combatants a non-combatants
    \item V případě že pustím do sítě kyber-zbraň, je téměř nemožné zaručit že bude napdat pouze vojenské cíle, ne civilní (nemocnice atd.)
    \item Vyvstává otázka zda data jsou vůbec vojenské cíle
    \item Pokud nejsou, je útok na data porušení ženevské konvence o útoku na civilní cíle?
    
\end{itemize}