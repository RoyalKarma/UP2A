% Povinný argument: Kód předmětu
\newcommand{\subject}{BPC-UP2A}
% Povinný argument: Název předmětu
\newcommand{\subjectname}{Úvod do práva ICT 2}
% Povinný argument: Seznam autorů
\newcommand{\authors}{Karma, Bořek}
% Povinný argument: Seznam korektorů
\newcommand{\corrections}{Nikdo, protože neumíme česky}
% Nepovinný argument: Popis dokumentu
\newcommand{\docdesc}{Příprava na zkoušku}
% Nepovinný argument: Cílová skupina dokumentu
\newcommand{\docgroup}{Informační bezpečnost, FEKT VUT}
% Nepovinný argument: URL repozitáře nebo jiný odkaz na dokument
\newcommand{\docurl}{https://github.com/RoyalKarma/UP2A}

\input{shared}
\clearpage

\begin{document}

\pagenumbering{gobble}
\setcounter{page}{0}

\titulka{}

% Do jaké hloubky se vypíše obsah (1 - section, 2 - subsection, ...)
\setcounter{tocdepth}{2}
\tableofcontents
\clearpage

\pagenumbering{arabic}

\section{Kyberkriminalita a elektronické důkazy}

\subsection{Popište nároky kladené na zákonnost elektronického důkazu.}
Elektronický důkaz = jakýkoliv důkaz přenášen v digitální podobě. Emaily, dig. fotografie, IM historie, textové dokumenty, video a audio záznamy, provozní údaje ...

Nároky na zákonnost:\begin{itemize}
    \item Důkaz byl opatřen způsobem, který stanovuje/připouští zákon
    \item Důkaz opatřen a proveden oprávněným procesním subjektem
    \item Neoprávněné získání - absolutní(neodstranitelná)/ relativní(odstranitelná) neúčinnost
\end{itemize}
Legální evidence na základě nelegálního důkazu je nepřípustná\\
Distributivní x nedistributivní právo
\begin{itemize}
    \item Standard dokazování: \uv{Nejsou důvodné pochybnosti}
    \item Na zákonnost důkazů se pokoušel k ÚS odvolávat i David Rath. Proti němu byly použity odposlechy a proto byl chycen při činu, odvolával se na nezákonnost těchto důkazů, čímž by padly vvšechny důkazy na nich stojící, protože všechny by byly nelegálně získané. Nezákonnost odposlechů ale ÚS zamítl, Rath odsouzen byl.
    \item Elektronické důkazy mají velkou výtěžnost
    \item With great power comes great responsibility, Spiderman
    \item Důraz kladený na zákonnost je z toho důvodu, abychom nežili v 1948, kde by byl každý sledován z preventivních důvodů
    \item Na závažnější zásahy do soukromí je potřeba soudní příkaz (odposlech) nebo povolení státního zástupce (sledování), to má sloužit proti zneužívání práva, ale bohužel se tyto žádosti nekontrolují tak důkladně, jak by měly
    \item Aby byl odposlech legální, je potřeba k němu vést protokol
\end{itemize}


\subsection{Vysvětlete strukturu §230 TZ (Neoprávněný přístup k počítačovému systému a nosiči informací). Uveďte příklady k jednotlivým odstavcům.}
\begin{itemize}
    \item Odstavec 1
          \begin{itemize}
              \item \textit{\uv{Kdo překoná bezpečnostní opatření, a tím neoprávněně získá přístup k počítačovému systému nebo k jeho části...}}
              \item V podstatě se jedná o prolomení důvěrnosti (CIA triáda = důvěrnost, integrita, dostupnost)
              \item Počítačový systém je chápán velmi široce (server, PC, kamera, telefon, router, webkamera...)
              \item Nelegální je prolomení jakéhokoli bezpečnostního opatření a nezáleží na síle bezpečnostního opatření (heslo, firewall, šifrování)
              \item Příklad: Použiju WiFi password cracker a v tu chvíli mě švestky berou
          \end{itemize}
    \item Odstavec 2
          \begin{itemize}
              \item \textit{\uv{Kdo získá přístup a data neoprávněně užije NEBO data neoprávněně vymaže, zničí, poškodí, změní, potlačí, sníží jejich kvalitu NEBO data padělá a pozmění tak, aby byla považována za pravá NEBO neoprávněně vloží data do počítačového systému...}}
              \item Jedná se o prolomení integrity a/anebo dostupnosti z CIA
              \item Neřeší se, jestli bylo překonáno opatření, za toto může být stíhán, i člověk, který měl oprávněný přístup, ale zneužil ho ke škodě
              \item Příklad: Jsem ajťák, šéf mi řekl, že na konci měsíce končím, tak mu složím databázi a už mě švestky berou
          \end{itemize}
    \item Odstavce 3-5 udávají výši trestů
          \begin{itemize}
              \item Všechny tresty jsou odnětí svobody
              \item Přitěžující okolnosti jsou rozsah škody, vlastní zisk, přítomnost v organizované skupině nebo pokud je takový útok specificky mířený na podnik nebo veřejnou správu
          \end{itemize}
\end{itemize}


\subsection{Popište rozdíly ve fungování §88 TŘ a §8 odst. 5 TŘ ve vztahu k mobilnímu telefonu.}
\begin{itemize}
    \item §8 odst. 5 - povinnost součinností
          \begin{itemize}
              \item Požadujeme sdělení informací od zprostředkovatelské služby - Státní orgány...
              \item můžeme požadovat logy, metadata
              \item Toto právo lze vynucovat soudem
              \item lze také ověřit hovory a SMS v telefonu, který byl policií již zajištěn, ale ne zprávy, které přicházejí až po zabavení, na ty už je potřeba aplikovat odposlech\\
          \end{itemize}

    \item §88 TŘ
          \begin{itemize}
              \item Může být vydán příkaz k odposlechu a záznamu telekomunikačního provozu, pokud lze důvodně předpokládat, že jím budou získány významné skutečnosti pro trestní řízení a nelze-li sledovaného účelu dosáhnout jinak nebo bylo-li by jinak jeho dosažení podstatně ztížené
          \end{itemize}
    \item Rozdíly
          \begin{itemize}
              \item §88 TŘ - menší pravděpodobnost porušení práv
              \item §8 v případě potřeby pouze metadat a přímého kontaktu s poskytovatelem
          \end{itemize}
\end{itemize}


\subsection{Vysvětlete, jaké elektronické důkazy mohou hrát roli při vyšetřování trestného činu vraždy podle §140 TZ.}
\begin{itemize}
    \item Máme materiální a formální pravdu. Důkazy slouží k nalezení formální pravdy, materiální pravda je nedosažitelná, pokud to vyšetřovatel neviděl na vlastní oči, protože neumíme cestovat časem
    \item Práce s důkazy
          \begin{itemize}
              \item Volné hodnocení důkazů - Soudce sám přisuzuje důkazům váhu a pravdivost
              \item Zákonná teorie důkazní - Rigorózní pravidla
          \end{itemize}
    \item Důkazy shromažďují orgány činné v trestním řízení (OČTŘ)
    \item Shromažďování důkazů
          \begin{itemize}
              \item Součinnost (Sdělte nám hento a tento)
              \item Povinnost k vydání věci (Hmotná věc, ale data následují osud věci, takže telefon i s daty)
              \item Odposlech (Vztahuje se např. i na data, která přijdou na telefon až po jeho zabavení, potřeba soudní příkaz nebo povolení účastníka odposlouchávané stanice)
              \item Záznam telekomunikačního provozu (Metadata, tzv. Data Retention, náhrady provozovatelům sítě)
              \item Sledování osob a věcí (Povolení státního zástupce, možnost při sledování dělat záznamy), nelze použít na telekomunikační provoz, ale jen na uložená data
          \end{itemize}
    \item Každý důkaz musí být pořízen přiměřeně k závažnosti trestného činu, protože se jedná o zásah do soukromí. Při vraždě by ale měly orgány mít možnost využít včech těchto 5 paragrafů.
\end{itemize}


\subsection{Popište, co je to data retention a vysvětlete zásadní milníky v jeho legislativní historii (rozhodovací praxe SDEU a ÚS).}

Pojem data retention označuje ukládání provozních a lokalizačních údajů u poskytovatelů telekomunikačních služeb, převážně pro účely vyšetřování trestné činnosti.\\
Zákon č. 127/2005 Sb., o elektronických komunikacích v § 97, odst. 3
\begin{itemize}
    \item Založen evropskou směrnicí (2006/24/ES)
    \item SDEU opakovaně judikoval, že plošné sledování odporuje evropskému právu i Chartě základních práv EU
    \item ÚS zrušil i u nás, zrušení povinnosti ukládání (Zákon o elektronických komunikacích a Trestní řád dostaly novelu, opravdu zrušeno)
    \item Poté nová úprava, data retention opět zavedeno, ale metadata přístupná jen pro některé trestné činy a uchováváno na kratší dobu
    \item ÚS nakonec došel k závěru, že česká právní úprava, která uchovávání dat přikazuje, je v pořádku
\end{itemize}

\section{Kybernetická bezpečnost}

\subsection{Jakými předpisy je upravována oblast kybernetické bezpernosti? Čemu se věnují?}
\begin{itemize}

\item \textbf{ZoKB a NIS} - věnují se udělování práv a povinností + chránění informačních aktiv(= cokolv co je nutno chránít)
\item \textbf{Zákon č.240/200sb., krizový zákon}
    \begin{itemize}
        \item Zákon stanovuje působnost a pravomoc státních orgánů a orgánů USC a práva a povinnosti fyzických a právnických osob při přípravě na krizové situace, kterou nesouvisejí se zajišťováním obrany České republiky před vnějším napadením
    \end{itemize}
    \item \textbf{Nařízení vlády č. 432/2012 Sb.,} o kritériích pro určení prvku kritické infrastruktury
    \item \textbf{Směrnice 2016/1148} o opatřeních k zajíštění vysoké společné úrovně bezpečností sítí a informačních systémů v Unii
    \item Komu stanovujeme povinnost?
    \begin{itemize}
        \item Soukromé objekty:
        \begin{itemize}
            \item Poskytovatel služby elektronických komunikací.
            \item Osoba zajišťující váznamnou síť.
        \end{itemize}
        \item Soukromé a veřejné subjekty:
        \begin{itemize}
            \item Správce IS/KS kritické informační infrastruktury.
        \end{itemize}
        \item Veřejné subjekty:
        \begin{itemize}
            \item Správce významného informačního systému
        \end{itemize}
        \item NIS: poskytovatel základní služby/digitální služby
    \end{itemize}
\end{itemize}

\subsection{Co je to "kritická infrastruktura", "kritická informační infrastruktura", "prvek kritické infrastruktury" a 
"provozovatel prvku kritické infrastruktury"? Jak spolu tyto pojmy souvisí?}
\begin{itemize}
    \item \textbf{Kritická infrastruktura}
    \begin{itemize}
        \item většina kritické infrastruktury je v soukromých rukou, i přesto že je důležitá pro chod státu a plnění jeho úloh.
        \item S tím spojená rizika (cíl pro paralyzaci státu) i povinnosti
        \item plyn,elektřina, voda, odpad...
    \end{itemize}
    \item Kritická informační struktura:
    \begin{itemize}
        \item kybernetická složka kritické infrastruktury (ovládací systémy...)
        \item ČR zabezpečuje NÚKIB, NCKB
        \item CERT týmy (Computer Emergency Response Team)
        \begin{itemize}
            \item Národní CERT- provozuje CZ.NIC; vládní CERT
            \item NÚKIB, vládní CERT (když mám problém na kritické infrastruktuře, nahlásím to zde), Národní CERT 
(cyber je citlivá oblast, subjektům pro stát důležité anonymizuje a posílá do vládního, provozuje 
CZ.NIC
        \end{itemize}
    \end{itemize}
\end{itemize}


\subsection{ Proč současná legislativa v oblasti kybernetické bezpečnosti nestanoví povinnosti individuálním uživatelům?}
\begin{itemize}
    \item Komu stanovujeme povinnost?
    \begin{itemize}
        \item Soukromé objekty:
        \begin{itemize}
            \item Poskytovatel služby elektronických komunikací.
            \item Osoba zajišťující váznamnou síť.
        \end{itemize}
        \item Soukromé a veřejné subjekty:
        \begin{itemize}
            \item Správce IS/KS kritické informační infrastruktury.
        \end{itemize}
        \item Veřejné subjekty:
        \begin{itemize}
            \item Správce významného informačního systému
        \end{itemize}
        \item NIS: poskytovatel základní služby/digitální služby
    \end{itemize}
    \item \textbf{Jde o takové subjekty které se podílí na významné}
    \begin{itemize}
        \item Nejsou to poskytovatelé obsahu, jednotliví uživatelé, ani provozovatele jiných služeb
        \item určité subjekty jsou do činnosti zapojeny v případě tzv. kybernetického nebezpečí (pouze po určitou dobu, vyhlášeno předsedou vlády)
    \end{itemize}
\end{itemize}


\subsection{ Jaký je vztah právní úpravy kybernetické bezpečnosti a právní úpravy ochrany osobních údajů?}
\begin{itemize}
    \item Na úrovní práva:
    \begin{itemize}
        \item porušení pracovněprávní povinnosti – vedoucí zaměstnanec kontroluje, že si nikam nepíšu heslo
        \item  porušení pracovněprávní povinnosti – vedoucí zaměstnanec kontroluje, že si nikam nepíšu heslo
        \item trestněprávní odpovědnost
        \item mezinárodní odpovědnost
    \end{itemize}
    \item stát má funkce dávat:
    \begin{itemize}
        \item distributivní práva (vlastnictví, soukromí)
        \item nedistributivní práva (bezpečnost (kybernetická))
        \item hledá se rovnováha: nedistributivní práva se konstruují omezováním distributivních
    \end{itemize}
\end{itemize}

\subsection{Vysvětlete, co je to analýza rizik a jak do ní vstupují varování vydaná NÚKIB.}
\begin{itemize}
    \item Vyhláška č. 316/2014 Sb.
    \item Analýza rizik - určuje případné bezpečnostní riziko související s užíváním prostředků, ke kterým se vztahuje varování (př. pokud používám SQLbackend v IS, musím mít ošetřený injection, před kterým NÚKIB varoval)
    \item \uv{Máme tady něco, před čím vyšlo varování NÚKIBu?} Pokud není compliant kritická infrastruktura, můžou být postihy.
    \item V kybernetické bezpečnosti se používají performativní pravidla (cíl, ne cesta)
    \item Compliance - soulad s pravidly bezpečnosti (Pokud je někdo compliant, splňuje, co má)
    \item Používá se Risk-Based Approach - Každý si riziko musí uvědomit sám a večerku není potřeba zabezpečovat tolik jako vojenskou základnu
    \item NÚKIB vydává varování podle již vyřešených incidentů a povinným subjektům stanoví lhůtu, do které musí implementovat jím navržené zvýšení bezpečnosti
\end{itemize}
//TODO mrknout sem znova, až budeme mít prezentaci k dispozici

\section{Identifikace, autentizace a datové schránky}
\subsection{Pojem elektronického podpisu a současné legislativní změny; rozdíl oproti elektronické pečeti.
}
\begin{itemize}
    \item  Elektronická formá právního jednání
    \begin{enumerate}
        \item Písemnost
        \begin{itemize}
            \item Elektronické prostředky nenahrazují právní jednání v písemné formě, jsou jejich jiným 
projevem stojícím paralelně vedle něj – jsou rovnocenné
        \end{itemize}
        \item Podpis jednajícího - cokoli co indetifukuje subjekt a je připojeno k dalším datům
        \begin{itemize}
            \item Podpis = Podpis / virtuální identita – emailová adresa, avatár, uživatelský účet, IP adresa, el. 
podpis, platnost od 2000, IDENTIFIKACE A INTEGRITA DOKUMENTU
            \item Oblast el. identifikace – certifikace, ověřování, zabezpečení, spolupráce států (Amerika X 
Evropa)
            \item stavěn na úroveň klasickému podpisu
            \item Nařízení eIDAS: \uv{data v elektronické podobě, která jsou připojena k jiným datům v elektronické podobě nebo 
jsou s nimi logicky spojena a která podepisující osoba používá k podepsání}
            \item Druhy: prostý, zaručený, kvalifikovaný el. podpis (certifikát)
            \item dokument podepisuje veřejný orgán, vždy nutnost podepsat kvalifikovaným el. podpisem 
            \item  V případě podepisování dokumentu soukromou osobou v případě komunikace s veřejným 
            orgánem – nutnost kvalifikovaného elektronického podpisu 
            \item Mimo výkon veřejné moci jakýkoliv podpis


        \end{itemize}
    \end{enumerate}
    \item Občanský zákoník (elektronická kontraktace
    \item Nařízení Evropského parlamentu a Rady (EU) č. 910/2014 ze dne 23. července 2014 o elektronické identifikaci 
a službách vytvářejících důvěru pro elektronické transakce na vnitřním trhu a o zrušení směrnice 1999/93/ES 
(eIDAS)
\item Zákon o elektronické identifikaci
\item Zákon o elektronických úkonech a autorizované konverzi dokumentů
\item Zákon o archivnictví a spisové službě
\item  Elektronická pečeť slouží jako důkaz toho, že elektronický dokument vydala určitá právnická osoba, a poskytuje jistotu o původu a integritě tohoto dokumentu. \textbf{Není spojena s konkrétní osobou.}
\end{itemize}

\subsection{Nařízení eIDAS – důvody přijetí}
\begin{itemize}
    \item eIDAS - Electronic indetification and services
    \item Nařízení Evropského parlamentu a Rady (EU) č. 910/2014 ze dne 23. července 2014 o elektronické identifikaci 
a službách vytvářejících důvěru pro elektronické transakce na vnitřním trhu a o zrušení směrnice 1999/93/ES 
\item Důvody:
\begin{itemize}
    \item  Dotvoření digitálního volného vnitřního trhu
    \item Zvýšení důveryhodnosti elektronických transakcí
    \item Vytvoření jednotného rámce pro elektronickou identifikaci
    \item Prokazování totožnosti v rámci EU
    \item Nahrazení současné legislativy pro elektronické podpisy
\end{itemize}
\end{itemize}



\section{Veřejnoprávní ochrana duševního vlastnictví}
\section{Ochrana osobních údajů}

\subsection{Ochrana soukromí a ochrana osobních údajů -- popište vztah mezi právní úpravou ochrany soukromí a osobních údajů}


\subsection{Základní principy a zásady zpracování osobních údajů}


\subsection{Osobní údaje a jejich zpracování -- vymezení pojmu a zúčastněné osoby}


\subsection{Povinnosti správce osobních údajů podle nařízení 2016/679}


\subsection{Práva subjektu údajů ve vztahu ke správci a zpracovateli podle nařízení 2016/679}

\section{Informace veřejného sektoru a Otevřená data}

\subsection{Pojem \uv{informace veřejného sektoru}}
\begin{itemize}
      \item Dava vs Informace(strukturovaná data, nesou význam, zakonodárci rozdíl neberou v potaz)
      \item Veřejná správa a samospráva generují velké množství takových dat
            \begin{itemize}
                  \item Statická data
                  \item Kartografická data
                  \item Meterologické údaje
                  \item Právní info
                  \item Jízdní řády
                  \item Katastr nemovitostí
            \end{itemize}
      \item 2 druhy přístupů: právo na informace (na vyžádání), opakované využití (dostupné stále)
      \item Právo na informace je ústavní právo (Všeobecná deklarace lidských práv)
      \item Informace veřejného sektoru slouží pro kontrolu činnosti veřejného orgánu (Právo na
            informace, Zákon 106/1999 Sb. – \uv{Vystošestkovat si to}; je to i obchodní artikl)
      \item Přístup k informacím veřejného sektoru je základní politické právo
      \item Princip publicity veřejné správy – Je nutné uveřejňovat veškeré informace a přístup lze
            odepřít pouze na základě zákona (Souvisí se zásadou legality – Správní orgán jedná jen
            v mezích zákona, rozdíl oproti soukromému právu)
      \item Zákon 106/1999 Sb.:
            \begin{itemize}
                  \item Obecný předpis
                        \begin{itemize}
                              \item Speciální právní úprava pro (tady se 106 neaplikuje):
                              \item Právo na informace o životním prostředí
                              \item Katastr nemovitostí
                              \item Živnostenský zákon…
                        \end{itemize}
                  \item Implementace evropské PSI směrnice (o opakovaném použití informací veřejného
                        sektoru)
                  \item Otázky: Kdo? Jaké informace? Opravné prostředky?
            \end{itemize}
      \item Poskytování informací na žádost nebo zveřejněním
            \begin{itemize}
                  \item  Na žádost:
                        \begin{itemize}
                              \item Žádost nemá přesně formalizovanou formu, písemně, ústně (\uv{Pls, na základě
                                          zákona číslo 106/1999 chci info…})
                              \item Poplatky za poskytnutí nesmí přesáhnout náklady pro zpřístupnění
                        \end{itemize}
                  \item Zveřejněním
                        \begin{itemize}
                              \item Povinné (Kdo je povinný subjekt: paragraf 5 Zák. 106/1999)
                              \item Dobrovolné
                              \item Co nejvíc strojově zpracovatelné a znovu užitelné (strojově čitelné, otevřený
                                    formát (lze číst softwarem přístupným všem), otevřená formální norma
                                    (pravidla pro strojovou interoperabilitu))
                        \end{itemize}
            \end{itemize}
\end{itemize}


\subsection{Povinné subjekty podle zákona č. 106/1999 Sb.}
\begin{itemize}
      \item Řeší se v paragrafu 2
      \item \textbf{státní orgány, územní samosprávné celky a jejich orgány a veřejné instituce
      \item subjekty, kterým zákon svěřil pravomoc rozhodovat o právech a povinnostech osob, v rozashu výkonu této pravomoci}
      \item V paragrafu 5 se pak řeší, kdo musí informace zveřejnit(subjekty na základě zákona vedou registry, evidence, seznamy, rejstříky, které jsou na základě zákona přístupné)
      \item Veřejné instituce- řeší se v novele, cíl zajistit aplikaci na veřejnoprávní média
            \begin{itemize}
                  \item Z této novely podle rozsudku soudů vyšlo najevo, že povinné subjekty jsou i  Všeobecná zdravotní pojišťovna a fond národního majetku( řešeno ÚS)
                  \item \textbf{Státní podnik Letiště Praha} - z tohoto rozhodnutí se derivoval test pro další rozhodovací praxi
                  \item \textbf{ČEZ} - nejvyšší správní soud řekl : \uv{yup} ÚS řekl:\uv{nope, výklad je příliš široký, je to osoba soukromého práva, i když má stát většinový podíl}
            \end{itemize}
      \item Novela: \textbf{a) Státní orgán, b) ůzemní samosprávní celek, c) právnická osoba zřízená zákonem, d) právnická osoba (kde:} zřizovatel =stát \textbf{nebo} zřízená pro uspokojení veřejného zájmu \textbf{nebo} financovaná převážně státem/samosprávným celkem/právnickou osobou zřízenou zákonem), \textbf{e)veřejní podnik} poskytují informace vztahující se k jejich činnosti.
\end{itemize}


\subsection{Pojem \uv{otevřená data}}
\begin{itemize}
      \item Způsob poskytování informací veřejného prostoru
      \item Mají být:
            \begin{itemize}
                  \item Úplná
                  \item Snadno dostupná
                  \item Strojově čitelná
                  \item Používající standardy s volně dostupnou specifikací
                  \item Zpřístupněná za jasně definovaných podmínek užití dat s minimem omezení
                  \item Dostupná uživatelům při vynaložení minima možných nákladů
            \end{itemize}
      \item 5 stupňů otevřenosti (příklady formátů v těchto stupních: pdf/xls/json/REST API/linked data)
      \item V Londýně 500+ apps postavených na otevřených datech, investice 1 mil. Lb, návratnost 58
            mil. Lb
      \item Problém „dokopat“ ke zveřejnění, protože instituce nechtějí investovat do zveřejnění, když za
            tím nevidí vidinu zisku (nemají jistotu, že je někdo využije), programátoři nemohou stavět
            aplikace, protože neví, v jakých formátech se budou data zveřejňovat


\end{itemize}


\subsection{Obecná právní úprava otevřených dat}
V ČR
\begin{itemize}
      \item Zákon č. 106/1999 Sb., o svobodném přístupu k informací
      \item \uv{OD novela} z. č. 298/2016 Sb. – navazuje na PSI novelu
            \begin{itemize}
                  \item kvalifikovaný způsob poskytování \textbf{zveřejněním} (OD novela stošestky)
                  \item informace zveřejňované způsobem umožňujícím dálkový přístup v otevřeném
                        a strojově čitelném formátu, jejichž způsob ani účel následného využití není omezen
                        a které jsou evidovány v \textbf{národním katalogu otevřených dat} (aspoň 3 stupně otevřenosti)
            \end{itemize}
      \item \textbf{Povinná otevřená data - }údaje ze systému ARES, jízdí řády, metadata registru smluv, údaje z IS o veřejných zakázkách, dotace
      \item \textbf{Doborvolná}- pokud není zákonem vyloučeno, možno poskytnou jakékoli informace
      \item Máme národní katalog otevřených dat ( ministerstvo vnitra), metadata ke všem OD, navázanost na Evropský katalog OD
      \item Dálkový přístup k OD
      \item Data z registrů se anonymizují(bez jmen, přijmení, data narození = pouze rok... Jinak jen v OD dotací - zde je úprava speciální)
\end{itemize}
EU
\begin{itemize}
      \item PSI směrnice - zrušená, pořád implementovaná v ČR
      \item Nová open data směrnice, Česko ještě neprovedlo transpozici
            \begin{itemize}
                  \item Novinky: veřejné subjekty – nově i veřejné podniky, dynamická data, datové sady
                        s vysokou socioekonomickou hodnotou, vyloučení zvláštních práv pořizovatele
                        databáze, dopadá na údaje z výzkumu
            \end{itemize}
\end{itemize}


\subsection{Právní překážky při poskytování informací a otevírání dat}
Právo na informace x právo na ochranu osobních údajů
\begin{itemize}
      \item Základní registry se sice zveřejňují, ale je potřeba anonymizovat
      \item Platy z veřejných prostředků: v zásadě poskytovat -> protesty
      \item Řešení: neuvádět jména, jen funkce
      \item Právo odmítnout poskytnout info o platu, pokud není splněno vše z:
            \begin{itemize}
                  \item účelem vyžádání informace je přispět k diskuzi o věcech veřejného zájmu
                  \item informace samotná se týká veřejného zájmu
                  \item žadatel o informaci plní úkoly či poslání dozoru veřejnosti či roli tzv \uv{Společenského hlídacího psa}
                  \item informace existuje a je dostupná
            \end{itemize}
      \item \uv{Hlídací pes} je problematický a proti němu se nejlépe ohrazuje
      \item Pro zpracování potřeba:
            \begin{itemize}
                  \item  Zákonný účet
                  \item Právní titul (plnění zákonné povinnosti nebo oprávněný zájem správce nebo třetí
                        strany)
            \end{itemize}
      \item Anonymizace - data by němala být deanonymizovatelná
\end{itemize}
Publikace otevřených dat x duševní vlastnictví
\begin{itemize}
      \item Dála jako součást datové sady
      \item Datová sada jako originál databáze
      \item Zvláštní práva pořizovatele databáze
      \item Pokud přítomné, jsou nutné licence
            \begin{itemize}
                  \item CC-3vrstvy(pro právníky, lidi, stroje)
                  \item V OD je nejvhodnější CC BY(uvést autora)
            \end{itemize}
      \item Duševní vlastnictví se nevztahuje na prostá data
\end{itemize}

\section{Výběr z přednášek}

\subsection{Předmět a účel právní regulace elektronických komunikací}
\begin{itemize}
    \item předměty regulace:
    \begin{itemize}
        \item Zajištění sítí el. komunikací
        \item Poskytnutí služeb el. komunikací
        \item provoz přístrojů
    \end{itemize}
    \item účely:
    \begin{itemize}
        \item Zajištění funkční hospodářské soutěže, konkurence a ochrany spotřebitelů
        \item Síťová neutralita
        \item Modernizace infrastruktury
        \item Ochrana soukromí a osobních údajů (ePrivacy
    \end{itemize}
\end{itemize}


\subsection{Síťová neutralita -- popis, právní úprava v EU, aktuální výzvy v USA}
\begin{itemize}
    \item Rovné podmínky přístupu k obsahu
    \item práv. úprava EU: nařízení č. 2015/2120 - umožněn zero rating (bezplatný přístup k internetu za určitch podmínek
    \item Zabraňuje poskytovatelům internetu zvýhodňovat nebo naopak zpomalovat/blokovat přístup na různé stránky nebo k různému obsahu
    \item USA : \begin{itemize}
        \item  FCC (Federal communications commission) odvolala pravidlo, které zaručuje net neutrality
        \item spor Mozilla vs FCC (2/2018) o zrušení rozhodnutí o zrušení síťové neutrality
        \item 9/2018 Přijetí California Internet Consumer Protection and Net Neutrality
        \item 10/2019 - rozhodnutí federálního odvolacího soudu - FCC mohla zrušit net neutrality, ale jednotlivé státy si mohou přijmout vlastní legislativu
        \item 7/2021 Biden doporučil FCC obnovení pravidel zajišťující net neutrality
    \end{itemize}
\end{itemize}


\subsection{Základní registry -- popis, účel a právní úprava}
\begin{itemize}
    \item  Účelem je zefektivnění a využítí možností současných technologií pro online přístupy kdykoliv a odkudkolive
    \item Omezení nezbytného sběru dat a informací a jejich efektivní předávání
    \item Zákonná úprava z. č. 111/2009 Sb., o základních registrech
    \item 4 základní:
    \begin{itemize}
        \item registr obyvatel\begin{itemize}
            \item referenční údaje o fyzických osobách
            \item spravuje ministerstvo vnitra a PČR
            \item info o občanech ČR, cizincích s trvalým pobytem, cizinci, jimž byl udělejn azyl, jiné fyz. osoby u nichž zákon stanovuju že budou uvedeny v registru obyvatel
        \end{itemize} 
        \item registr osob \begin{itemize}
            \item  evidence právnických osob, organizačních složek státu, organizací s mezinárodním prvkem, podnikajících fyzických osob
            \item Obsahuje základní údaje o osobách
            \item využívají všechny orgány veřejné správy, které mají k tomuto oprávnění
        \end{itemize}
        \item registr územní identifikace\begin{itemize}
            \item Katastrální úřad, Stavební úřad, Obce
            \item info o územních prvcích, adresách, územních identifikací, územně evidenčních jednotkách
        \end{itemize}
        \item registr práv a povinností \begin{itemize}
            \item Ministerstvo vnitra
            \item Zdroj údajů pro informační systémy zákl. registrů při řízení přístupu uživatelů k údajům v jednotlivých registrech
            \item referenční údaje o právech a povinnostech osob
        \end{itemize}
    \end{itemize}
\end{itemize}

\subsection{Hlavní překážky pro zavedení komplexní eJustice v ČR}
\begin{itemize}
    \item Elektronizace soudnictví, využítí ICT v justici (neplést s Online Dispute Resolution)
    \item Problémy
    \begin{itemize}
        \item Odpor justičního personálu
        \item Stereotypy, návyky
        \item Nekoncepčnost
        \item Zabezpečení
        \item Kompatibilita - elektronické spisy musí být jednotné, aby měly smysl
        \item Problém i s využitím elektronického spisu, soudy si často vedou dvojí evidenci, papírovou a elektronickou, přičemž papírová for some reason je ta primární
        \item Plně elektronický spis v našem soudnictví: centrální elektronický platební rozkaz, funguje pomocí datových schránek nebo hybridní pošty (konvertuje digitální na papírové dopisy)
        \item Neexistuje samostaný cíl eJustice, jen je to strategický rámec rozvoje veřejné správy
        \item Jedná se o zásah do soudcovské nezávislosti, protože se musí učit nové věci?
        \item Je povinné splnit podmínku rovného přístupu, ale není zaručeno, že každý disponuje elektronickým přístupem
        \item např. eISIR (elektronický insolvenční rejstřík) ztroskotal, protože ministerstvo spravedlnnosti nedokázalo zadat požadavky dostatečně tehnicky přesně
    \end{itemize}
\end{itemize}

\subsection{Co je to princip distinkce a jak se vztahuje ke kyberzbraním a kyber-válce?}
\begin{itemize}
    \item Mezinárodní právo
    \item Problém dělání rozdílů mezi combatants a non-combatants
    \item V případě že pustím do sítě kyber-zbraň, je téměř nemožné zaručit že bude napadat pouze vojenské cíle, ne civilní (nemocnice atd.)
    \item Vyvstává otázka, zda data jsou vůbec vojenské cíle
    \item Pokud nejsou, je útok na data porušení ženevské konvence o útoku na civilní cíle?
    
\end{itemize}

\end{document}
